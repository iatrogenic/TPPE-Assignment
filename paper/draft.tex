\documentclass[]{article}

\usepackage{amsmath, amssymb, amsthm}

%custom commands
\newcommand{\RR}{\mathbb{R}}
\newcommand{\NN}{\mathbb{N}}
%Probability
\newcommand{\EE}{\mathbb{E}}
\newcommand{\PP}{\mathbb{P}}



\newtheorem{theorem}{Theorem}
\newtheorem{proposition}{Proposition}

\theoremstyle{definition}
\newtheorem{definition}{Definition}[section]

\theoremstyle{remark}
\newtheorem*{remark}{Remark}

%opening
\title{TPPE assignment (DRAFT)}
\author{Maria Jerónimo Martins \\ Martim Roberto Alves da Costa}

\begin{document}

\maketitle

\section{Introduction}
There is a class of financial instruments, called \textit{derivatives}, whose value is dependent on the value of another security (typically stock), called the \textit{underlying security}, or simply \textit{underlying}. The most well-known example of derivatives are stock options, which split into \textit{call} and \textit{put} options. The former grants the right to purchase the underlying stock at a fixed price $K$, called the \textit{strike price}, at a specified date, called the \textit{expiration date}.\footnote{To be specific, this is a description of an \textit{European} call option.}
	A put option grants the right to sell the underlying at an agreed upon strike price and expiration date. 
	\par One of the fundamental problems in mathematical finance is the determination of the \textit{fair value} of such financial instruments. This problem, called the \textit{deriative pricing problem}, is explored in this paper for European options with the purpose of applying the Lebesgue integral.

\section{Pricing an European Option}
Consider an option with strike price $K$, expiration date $T$, and denote the spot price of the underlying stock at time $t \in [0, T]$ by $S(t)$. Its price at time $t$ will be denoted by $f(S(t))$. In general, when the expiration date arrives, it only makes sense to exercise an option, if and only if, there is a positive return. The \textit{payoff} functions for the buyer of an option are
\begin{center}
	\begin{tabular}{|c|c|}
		\hline
		Call      & Put       \\ \hline
		$(S-K)^+$ & $(K-S)^+$ \\ \hline
	\end{tabular}
\end{center}

where $X^+ = \max \{X,0\}$.

\subsection{Arbitrage}
\par
An \textit{arbitrage opportunity} is an investment opportunity that is guaranteed not to result in a loss and may (with positive probability) result in a gain. The fact that markets adjust to eliminate arbitrage leads to the fundamental principle of asset pricing: \textit{it only makes sense to price securities under the assumption that there is no arbitrage}. This is called the \textit{no-arbitrage principle}.
\par 
An elementary implication of this principle is that two portfolios, $A$ and $B$, with the same payoff function, must, at time $t=0$, be equally valuable
\[\mathcal{V}_{A}(0) = \mathcal{V}_B (0). \]
This means that if we construct a portfolio whose initial value is known and whose payoff function equals that of our European option, then we can consequently also price the option itself. Such a portfolio is known as a \textit{replicating portfolio}.
\par We now apply the previous corollary to derive a relationship between the prices of a put and call option. Suppose that a stock is currently selling at a price of $S(0)$ per share. Let $P$ and $C$ denote the price of a European put and call, respectively, both with the same strike price $K$ and expiration date $T$. Let also $d(0)$ represent the present value of any dividends paid by the stock during the period in question and $r$ denote the risk-free rate. A portfolio consisting of a put and a call will have initial value equal to $C-P$ and a payoff at time $T$ equal to
\[(S(T) - K)^+ -(K-S(T))^+ = S(T)-K.\]
Now consider a portfolio consisting of a share of the aforementioned stock and a debt of $x$ dollars, so that its initial value is $S(0)-x$ and payoff at time $T$ equal to
\[  S(T) - x e^{rT} +d(0) e^{r T}. \]
By equating the two payoff functions, we obtain that
\[
S(T) - x e^{rT} +d(0) e^{r T} = S(T) - K,
\]
which implies that $x=K e^{-rT} + d(0)$. With this value for $x$ the two payoffs are equal, and we conclude that the initial values must also be identical, i.e.
\[ C-P = S(0) - Ke^{-rT} - d(0). \]
This last equation is called the \textit{put-call parity formula}.


\newpage
\nocite{*}
\bibliographystyle{plain}
\bibliography{bibliography} 

\end{document}

